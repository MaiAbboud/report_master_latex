\section{تعريف مفهوم الملاحقة
\textLR{\cite{Abbass21}}
\textLR{\cite{Zhang21}}
}
نهتم في بحثنا بملاحقة غرض واحد
\textLR{single object tracking}
  دون معلومات مسبقة عن هذا الغرض 
\textLR{general object tracking}.
  في هذه الحالة نعرف عملية الملاحقة بأنها تقدير لحالة الغرض المراد ملاحقته في الصور المتلاحقة للفيديو بالاعتماد فقط على مظهره في الإطار الأول. يمكن التعبير عن حالة الغرض بمستطيل محيط به في كل إطار من إطارات الفيديو، ندعوه بالـ مستطيل المحيط
\textLR{bounding box}،
يتم تحديده يدويا في الإطار الأول. ونسمي ما بداخل المستطيل المحدد يدويا في الإطار الأول بالـ
\textLR{template}
 قالب.
بعد تحديد الـ
\textLR{template}
 في الإطار الأول للملاحقة يأتي إطار جديد من الفيديو والمهمة هي تقدير المستطيل المحيط. من أجل الإطار الجديد نقوم باقتطاع المنطقة المحيطة بموقع الهدف في الإطار السابق بحجم معين ( مثلا أربع أضعاف المستطيل المحيط السابق) ونسمي هذه المنطقة بنافذة البحث 
\textLR{search window}،
لأننا لا نبحث عن الهدف في كامل الصورة ولكن فقط في هذه المنطقة.
%معظم الملاحقات تتبع إطار العمل المشروح سابقاً والموضح في الشكل التالي:
%
%صورة توضيحية ****
