\chapter{الخاتمة والآفاق المستقبلية}
\section{الخاتمة}
قمنا في هذا البحث بالتعرف على أنواع الملاحقة، وبالتركيز على الملاحقات التي تستخدم تقنيات التعلم العميق، وبالتعرف على نموذج المحول وعلى التابع الأساسي فيه وهو تابع الانتباه، كما أننا ذكرنا تطبيقات المحول في مجال الرؤية الحاسوبية، وركزنا على تطبيقاته في مجال الملاحقة.
وبما أن النموذج الأساسي في البحث هو 
\textLR{SwinTrack}،
والذي يستخدم محول 
\textLR{Swin}
لاستخلاص السمات، لذلك قمنا بشرح مفصل عن هذا المحول، وسبب ملائمته لتطبيقات الرؤية الصنعية وللزمن الحقيقي.
ثم شرحنا عن النموذج الأساسي
\textLR{SwinTrack-Tiny} 
وسبب اختيارنا له، وكونه يستخدم فقط المعلومات المكانية، لذلك كانت فكرة التعديل هي باستخدام المعلومات الزمانية، وذلك عن طريق إدخال صورة الغرض الجديدة 
\textLR{updated template}
كدخل ثالث إلى النموذج، ودمج سمات الصورة الجديدة بسمات الـ
\textLR{template} 
الابتدائي عن طريق كتلة انتباه، وذلك للحفاظ على أبعاد النموذج الأصلي للاستفادة من أوزانه.
تم تجريب كتلة انتباه بـ
$4$
رؤوس، وبتدريب هذه الكتلة الإضافية فقط مع الاستفادة من أوزان النموذج الأصلي تحسن الأداء عند اختبار النموذج على معطيات الاختبار الخاصة بمجموعة المعطيات
\textLR{GOT-10k}.
التعديل الثاني الذي قمنا به هو استخدام $8$ رؤوس في كتلة الانتباه، أيضا كان هناك تحسن مقارنة بالنموذج الأساسي.
وأخيراً عرضنا نتائج الـ
\textLR{evaluation}
ونتائج الاختبار وقارنا بين النموذج الأصلي وبين النموذجين المعدلين، وعرضنا بعض من النتائج التجريبية على صور من فيديوهات حقيقية.
\section{الآفاق المستقبلية}
يمكن تطبيق العديد من الأفكار واختبار فعاليتها على خوارزميات الملاحقة مثل :
\begin{itemize}
	\item
	تجريب تدريب شبكتي التصنيف والـ
	\textLR{regression}
	وذلك بتجميد أوزان النموذج الباقية بما فيها كتلة الانتباه المدربة في التجربة الأولى.
	\item
	تجريب التدريب على معطيات تسلسلية كون التدريب يتم باختيار الصور بشكل عشوائي من مجموعة التدريب، ودراسة تأثير طريقة التدريب على الأداء. وذلك لأن مسألة الملاحقة تختلف عن مسألة الكشف، وبالتالي من الأفضل أن تستفيد خوارزمية الملاحقة من تسلسل الإطارات لتحديد مكان الغرض.
	\item
	تجريب ترميز مكاني مختلف لكل من صورتي الهدف لتمييز مصدر السمات كما في النموذج الأساسي.
	
\end{itemize}
%\section{صعوبات المشروع}
%واجهنا العديد من الصعوبات في البحث وكان من بينها
%\begin{itemize}
%	\item
%	صعوبة تجهيز بيئة العمل، إذ يتطلب ذلك حاسوب مجهز بـ
%	\textLR{GPU}
%	تقبل تنصيب مكتبات 
%	\textLR{CUDA}.
%	الحاسب المتوفر لدينا مواصفاته 
%	\textLR{core-i9 NVIDIA GeForce RTX 2080Ti},
%	يقبل تنصيب مكتبة 
%	\textLR{CUDA}.
%	\newline
%	أيضا نحتاج إلى 
	
	
%\end{itemize}