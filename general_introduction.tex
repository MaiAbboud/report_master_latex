\chapter{مقدمة عامة}
\section{تمهيد}
ملاحقة الأغراض
\textLR{visual object tracking}
هي مجال من مجالات الرؤية الحاسوبية
\textLR{computer vision}،
والتي تهتم بتحديد حالة الغرض المراد ملاحقته في كل إطار من إطارات الفيديو. وهي موضوع بحث نشط لما تواجهه من الكثير من التحديات والمشكلات والتي تجعل من عملية الملاحقة مهمة صعبة، بالإضافة إلى تطبيقاتها في مجالات متنوعة مثل القيادة الذاتية، الطب، الرياضة، المجال العسكري، الأمن، الترفيه، وغيرها.


ظهر في العقود الأخيرة كم كبير من خوارزميات الملاحقة أبرزها
\textLR{NCC, Meanshift, MDNet, MOSSE}
وغيرها، والتي تستخدم تقنيات مختلفة لحل مشاكل الملاحقة. وبسبب تنوع الأساليب والتقنيات وكثرتها قررنا أن نركز في هذا البحث على خوارزميات الملاحقة التي تستخدم تقنيات التعلم العميق وذلك لعدة أسباب:
\begin{itemize}
	\item
تقنيات التعلم العميق هي الأكثر استخداماً في أبحاث الملاحقة في السنوات الأخيرة.
	\item
وجود العديد من معطيات التدريب الكافية لتدريب نماذج التعلم العميق مثل
\textLR{GOT-10k\cite{got10k}},
\textLR{LaSOT\cite{Lasot}}
		وغيرها.
	\item
تتصدر  ملاحقات التعلم العميق المراتب الأولى من حيث الأداء مقارنة بالملاحقات التقليدية.
	\item
ظهور العديد من ملاحقات التعلم العميق سريعة الأداء بشكل كافي لتطبيقات الزمن الحقيقي.
\end{itemize}
ظهرت في السنتين الأخيرتين بعض خوارزميات الملاحقة التي تستخدم المحول 
\textLR{Transformer\cite{Vaswani17}}
في بنيتها مثل
\textLR{STARK\cite{Stark}},
\textLR{SwinTrack\cite{swinTrack}},
وقد تفوقت في الأداء على خوارزميات الملاحقة الأخرى وقت ظهورها، بالإضافة إلى تحقيقها لمتطلبات الزمن الحقيقي في حال تشغيلها على وحدة معالجة رسومات 
$GPU$
كافية.
وبما أن المحول أعطى نتائج واعدة في مجال الرؤية الصنعية كما في خوارزمية الكشف 
\textLR{DETR\cite{DETR}}
فقد اخترنا في بحثنا التركيز على خوارزميات الملاحقة التي تستخدم المحول.
\newline
يسعى هذا البحث إلى التعرف على خوارزميات الملاحقة الحديثة بشكل عام، وإلى تقديم دراسة تفصيلية عن نموذج المحول، وتطوير نظام ملاحقة يعتمد في بنيته على المحول، مع الأخذ بعين الاعتبار سرعة الأداء.
\section{إشكالية ودوافع البحث}
ماتزال الملاحقة تواجه الكثير من التحديات والعقبات والتي تجعلها من المجالات الصعبة في الرؤية الحاسوبية.
إذ يتم تقييم خوارزميات الملاحقة من خلال قدرتها على تتبع الغرض الملاحَق وتجاوز التغيرات التي يعاني منها الغرض أو البيئة المحيطة به.


في هذه الفقرة سنذكر عدة مشكلات تواجه الملاحقة :
\textLR{\cite{Abbass21}}
\textLR{\cite{Zhang21}}
\textLR{\cite{got10k}}
\begin{itemize}
	\item
	تغيير الإضاءة.
	\item{الاحتجاب 
		\textLR{occlusion}
		: يمكن أن تحجب الخلفية جزء أو كامل الغرض.}
	\item
اختفاء الغرض كليا من المشهد.
	\item
تغير شكل الغرض بسبب طبيعته المفصلية.
	\item
تغير شكل الغرض بسبب تغير زاوية الرؤية.
	\item
حركة الكاميرا.
	\item{حركة الهدف إما بشكل سريع أو بشكل مفاجئ أو بسبب الدوران.}
	\item{وجود أغراض مشابهة للغرض المراد ملاحقته، أو وجود خلفية لها نفس بنية الغرض، وهذا ما يجعله يتماهى مع الخلفية, وبالتالي تزداد صعوبة الملاحقة.}
	\item{
		عدم وجود تفاصيل في الصورة إذ تكمن الصعوبة هنا في الحصول على عدد كاف من السمات لتوليد نموذج لمظهر الغرض 
		\textLR{appearance model}}.
	\item
	ملاحقة أغراض صغيرة.
	\item
	دقة الكاميرة المستخدمة.
	\item
		الملاحقة بالزمن الحقيقي، إذ تتطلب بعض التطبيقات أن تكون الملاحقة بأداء عالي وبسرعة كافية لمعالجة الصور بالزمن الحقيقي.
\end{itemize}
بالإضافة إلى التحديات المذكورة سابقاً فإن معظم الملاحقات الحديثة تستفيد فقط من معلومات مظهر الغرض والمنطقة المحيطة به أي المعلومات المكانية
\textLR{spatial information},
وتتجاهل المعلومات الزمانية
\textLR{temporal information}.
إذ أنها تعامل مسألة الملاحقة وكأنها مسألة كشف للأغراض دون الاستفادة من الإطارات المتلاحقة أو معلومات تغير المظهر.
\newline
الإشكالية الأخرى الناتجة عن اختيارنا لخوارزمية ملاحقة تعتمد على  التعلم العميق هي زمن التدريب الكبير للنموذج في حال عدم توفر عتاد صلب ذو إمكانيات قوية، بالإضافة إلى الحجم الكبير لمعطيات التدريب الخاصة بمسألة الملاحقة، فمثلاً حجم الملف الخاص بمجموعة المعطيات 
\textLR{GOT-10k\cite{got10k}}
يصل إلى أكثر من $60GB$.
\section{فكرة الحل المقترح}
من النقاط الأساسية التي يعالجها هذا البحث هي تطوير إحدى خوارزميات الملاحقة التي تعتمد فقط على المعلومات المكانية وتستخدم المحول وتحقق متطلبات الزمن الحقيقي وهي خوارزمية
\textLR{SwinTrack\cite{swinTrack}},
وذلك عن طريق الاستفادة من المعلومات الزمانية الناتجة عن تغير مظهر الغرض.
يتم ذلك عن طريق تطبيق دخل ثالث إلى الخوارزمية وهو صورة الغرض الجديدة
كما في خوارزمية 
\textLR{STARK\cite{Stark}},
ودمج سمات هذه الصورة مع سمات صورة الغرض الإبتدائية المأخوذة من أول إطار في عملية الملاحقة.
يتم هذا الدمج عن طريق تابع الانتباه
\textLR{Attention}،
وهو التابع الأساسي في نموذج المحول. هذا الدمج يضمن المحافظة على أبعاد نموذج الملاحقة الأساسي، وبالتالي يمكن الاستفادة من الأوزان المدربة سابقاً، تجنباَ لتدريب النموذج بالكامل، ونكتفي بتدريب كتلة الانتباه الإضافية فقط بسبب الإمكانيات المحدودة للعتاد الصلب المتوفر لدينا.
\section{المساهمات الأساسية في البحث}
الإضافات الجديدة التي يقدمها هذا البحث هي:
\begin{itemize}
\item
تطوير خوارزمية الملاحقة 
\textLR{SwinTrack\cite{swinTrack}}
عبر إدخال معلومات زمانية وهي صورة الغرض الجديدة كدخل ثالث.
\item
المحافظة على أبعاد النموذج الأساسي وذلك عن طريق استخدام كتلة انتباه لدمج سمات صورة الغرض الإبتدائية مع صورة الغرض الجديدة.
\item
تدريب كتلة الانتباه باستخدام مجموعة المعطيات الخاصة بمسألة الملاحقة
\textLR{GOT-10k\cite{got10k}}.
\item
اختبار إضافة كتلة انتباه بثمانية رؤوس وبأربعة رؤوس.
\item
تحسين الأداء مع المحافظة على متطلبات الزمن الحقيقي.
%\item
%وعلى المستوى العملي تعديل الكود الأساسي للخوارزمية بإضافة تابع لتجريب خوارزمية الملاحقة
%\textLR{SwinTrack\cite{swinTrack}}
%من أجل فيديوهات.
\end{itemize}
\section{مخطط البحث}
يتضمن هذا البحث عدة فصول، يبدأ الفصل الثاني بتعريف مسألة الملاحقة، ويحتوي على دراسة مرجعية لتصنيف خوارزميات الملاحقة التي تستخدم تقنيات التعلم العميق. ثم نقدم دراسة نظرية تفصيلية لنموذج المحول ولتابع الانتباه، وذكر تطبيقاته في مجال الرؤية الصنعية مثل الكشف والتصنيف ونخص بالذكر تطبيقاته في مجال الملاحقة، وأخيراً نقوم بشرح محول 
\textLR{Swin\cite{swintransformer}}،
وهو النموذج المعتمد لاستخلاص سمات الصور باعتباره
\textLR{backbone}.
\newline
أما في الفصل الثالث سنقوم بعرض النماذح الأساسية التي اعتُمِد عليها وهي خوارزمية 
\textLR{STARK\cite{Stark}}
وخوارزمية
\textLR{SwinTrack\cite{swinTrack}}،
ثم نعرض التعديل المقترح.
\newline
في الفصل الرابع نعرض نتائج التجارب التي قمنا بها، مع شرح عن مجموعة المعطيات الخاصة بالتدريب والاختبار،
وفي الفصل الخامس والأخير نذكر الخاتمة والآفاق المستقبلية بالإضافة إلى الصعوبات التي واجهتنا في هذا المشروع.


